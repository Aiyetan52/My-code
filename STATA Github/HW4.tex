
\documentclass{article}
\usepackage{amsmath}

%%%%%%%%%%%%%%%%%%%%%%%%%%%%%%%%%%%%%%%%%%%%%%%%%%%%%%%%%%%%%%%%%%%%%%%%%%%%%%%%%%%%%%%%%%%%%%%%%%%%%%%%%%%%%%%%%%%%%%%%%%%%%%%%%%%%%%%%%%%%%%%%%%%%%%%%%%%%%%%%%%%%%%%%%%%%%%%%%%%%%%%%%%%%%%%%%%%%%%%%%%%%%%%%%%%%%%%%%%%%%%%%%%%%
%TCIDATA{OutputFilter=LATEX.DLL}
%TCIDATA{Version=5.50.0.2953}
%TCIDATA{<META NAME="SaveForMode" CONTENT="1">}
%TCIDATA{BibliographyScheme=Manual}
%TCIDATA{Created=Tuesday, September 24, 2019 12:30:33}
%TCIDATA{LastRevised=Tuesday, September 24, 2019 13:19:25}
%TCIDATA{<META NAME="GraphicsSave" CONTENT="32">}
%TCIDATA{<META NAME="DocumentShell" CONTENT="Standard LaTeX\Blank - Standard LaTeX Article">}
%TCIDATA{CSTFile=40 LaTeX article.cst}

\newtheorem{theorem}{Theorem}
\newtheorem{acknowledgement}[theorem]{Acknowledgement}
\newtheorem{algorithm}[theorem]{Algorithm}
\newtheorem{axiom}[theorem]{Axiom}
\newtheorem{case}[theorem]{Case}
\newtheorem{claim}[theorem]{Claim}
\newtheorem{conclusion}[theorem]{Conclusion}
\newtheorem{condition}[theorem]{Condition}
\newtheorem{conjecture}[theorem]{Conjecture}
\newtheorem{corollary}[theorem]{Corollary}
\newtheorem{criterion}[theorem]{Criterion}
\newtheorem{definition}[theorem]{Definition}
\newtheorem{example}[theorem]{Example}
\newtheorem{exercise}[theorem]{Exercise}
\newtheorem{lemma}[theorem]{Lemma}
\newtheorem{notation}[theorem]{Notation}
\newtheorem{problem}[theorem]{Problem}
\newtheorem{proposition}[theorem]{Proposition}
\newtheorem{remark}[theorem]{Remark}
\newtheorem{solution}[theorem]{Solution}
\newtheorem{summary}[theorem]{Summary}
\newenvironment{proof}[1][Proof]{\noindent\textbf{#1.} }{\ \rule{0.5em}{0.5em}}
\input{tcilatex}

\begin{document}


\begin{enumerate}
\item Find demand for good $y$ from the consumer's EMP

\begin{itemize}
\item To find demand for good $y$, we plug the demand for good $x$ into her
utility target $u=2x^{1/3}+y$

\[
u=2\left( \underset{x_{E}}{\underbrace{\left( \frac{2p_{y}}{3p_{x}}\right)
^{3/2}}}\right) ^{1/3}+y
\]

Solving for $y$, we obtain the consumer's demand for good $y$

\[
y^{E}(p_{x},p_{y},u)=u-2\left( \frac{2p_{y}}{3p_{x}}\right) ^{1/2}
\]
\end{itemize}

\item Calculate the CV for a price increase from $p_{x}=\$5$ to $%
p_{x}^{\prime }=\$10$, where $u=30$ and $p_{y}=\$1.$

\begin{itemize}
\item The CV becomes the integral of the demand function for good $x$
between prices $p_{x}=\$5$ and $p_{x}^{\prime }=\$10$, that is 

\[
CV=\int_{p_{x}}^{p_{x}^{\prime
}}x^{E}(p_{x},p_{y},u)dp_{x}=\int_{5}^{10}\left( \frac{2}{3p_{x}}\right)
^{3/2}dp_{x}
\]

\[
=0.54\int_{5}^{10}\left( \frac{1}{p_{x}}\right)
^{3/2}dp_{x}=0.54\int_{5}^{10}\left( p_{x}\right) ^{-3/2}dp_{x}
\]
\end{itemize}

The integral of $(p_{x})^{a}$ is $\frac{p_{x}^{a+1}}{a+1}$(power rule of
integration), implying that the integral of $(p_{x})^{-3/2}$ is $\frac{%
(p_{x})^{-3/2+1}}{\frac{-3}{2}+1}=\frac{(p_{x})^{-3/2+1}}{\frac{-1}{2}}%
=-2(p_{x})^{-1/2}$. Therefore, the above integral becomes

\[
CV=0.54\times |-2(p_{x})^{-1/2}|_{5}^{10}
\]

\[
CV=0.54\left[ -2(10^{-1/2}-5^{-1/2})\right] 
\]

\[
CV=-1.09\times \left( \frac{1}{\sqrt{10}}-\frac{1}{\sqrt{5}}\right) 
\]

\[
CV=0.14
\]

This means that we need to give the consumer \$0.14 to make her as well off
as she is before the price increase.

\item Calculate the CV of the above price change in good $x$, but using the
demand function of good $y$ to see how the consumer's welfare in her
purchases \ of good $y$ is affected by a more expensive good $x$

\begin{itemize}
\item The CV becomes the intergral of the demand function for good $y$
between prices $p_{x}=\$5$ and $p_{x}^{\prime }=\$10$, that is 

\[
CV=\int_{p_{x}}^{p_{x}^{\prime
}}y^{E}(p_{x},p_{y},u)dp_{x}=\int_{5}^{10}30-2\left( \frac{2}{3p_{x}}\right)
^{1/2}dp_{x}
\]
\end{itemize}

\[
\int_{5}^{10}30dp_{x}-1.63\int_{5}^{10}\left( p_{x}\right) ^{-1/2}dp_{x}
\]

The integral of $(p_{x})^{a}$ is $\frac{p_{x}^{a+1}}{a+1}$(power rule of
integration), implying that the integral of $(p_{x})^{-1/2}$ is $\frac{%
(p_{x})^{-1/2+1}}{\frac{-1}{2}+1}=\frac{(p_{x})^{1/2}}{\frac{1}{2}}%
=2(p_{x})^{-1/2}$. Therefore, the above integral becomes

\[
CV=|30p_{x}|_{5}^{10}-1.63\times |2(p_{x})^{1/2}|_{5}^{10}
\]

\[
CV=30\times |p_{x}|_{5}^{10}-3.26\times |(p_{x})^{1/2}|_{5}^{10}
\]

\[
CV=30(10-5)-3.26[10^{1/2}-5^{1/2}]
\]

\[
CV=146.98
\]

This means that we need to give the consumer \$146.98 to make her as well
off as she is before the price increase.
\end{enumerate}

\end{document}
